\documentclass[12pt, titlepage, oneside]{article}

\usepackage[margin=1in]{geometry}
\usepackage{siunitx, booktabs, amsmath, enumitem, pdfpages,mathrsfs,tabularx,caption, graphicx, pgfplots, textcomp,wrapfig, commath, svg, amsfonts, relsize}
\usepackage{parskip}
\usepackage[siunitx]{circuitikz}
\sisetup{detect-weight=true, detect-family=true}
\renewcommand{\vec}[1]{\oldvec{\bm{#1}}}
\renewcommand{\hat}[1]{\oldhat{\bm{#1}}}
\renewcommand{\b}[1]{\textbf{#1}}
\newcommand{\de}[1]{\noindent\fbox{\parbox{\textwidth}{#1}}}

\newcommand{\be}{\begin{align*}}
\newcommand{\ee}{\end{align*}}

\newcommand{\items}{\begin{itemize}}
\newcommand{\eitems}{\end{itemize}}


\newcommand{\ex}{Ex. }
\newcommand{\exe}{_}

\newcommand{\n}{\cap}
\renewcommand{\u}{\cup}


\begin{document}
	
	\textbf{ELECENG 3TQ3}\\
	\textbf{Elston A.}
	
\section{Lecture 6}

\subsection{Discrete Random Variables}

We assign number to the outcomes in sample space. Once we observe a number, we refer to that observation by a random variable.
Each random variable has a range we denote as $S_X$.



A probability model always begins with an experiment. Each random variable is directly related to this experiment. There are 3 types of relationships (how random variables can be related to your experiment):

\items
\item The random variable is the observation
\item The random variable is a function of the observation
\item The random variable is a function of another random variable
\eitems

\b{Random variable is the observation:} The experiment is to attach a photo detector to an optical fiber and count the number of photons arriving in a one microsecond time interval. Each observation is a random variable $X$. The range of $X$ is $S_X = \{0,1,2,\dots\}$. In this case $S_X$ and the range $X$ is identical.

A better example is if we wanted to determine the number of cars passing through a particular intersection at a particular time during a particular time interval range.

\b{Random variable is a function of an observation}: If we had a class of 30 students and wanted to determine the number of students who got an A- or better. Then the sample space would be the final grades, but the random variable is the number of students who achieved an A- or better. 

\b{Random variable is  is a function of another random variable}: number of new daily COVID cases is a random variable $X$, we introduce a new random variable $Y$ as estimated cost to OHIP. Obviously $Y$ depends on $X$ hence $Y=f(X)$. 
	
	
\b{Definition}: A random variable consists of an experiment with a probabilistic measure $P[\cdot]$ defined on a sample space $S$ and a function that assigns a real number to each outcome in the sample space of the experiment.

\subsection{Discrete vs. Continuous}

Example of a discrete random variable are the number of cars in a parking lot, or the number of lights in a room. Since we have a number for each value in the range $S_X$ as there cannot be $0.51323\dots$ of a car, we understand these variables as discrete.

Example of a continuous random variable is the amount of time it takes to travel from Square One to McMaster University. Since the range $S_X$ is time, time has an infinite range and is continuous. So if the range is continuous, then the random variable is continuous. 

\b{Definition:} $X$ is a discrete random variable if the range of $X$ is a countable set.

\b{Definition:} $X$ is a finite random variable if the range is a finite set.

\subsection{Probability Mass Function}

The probability mass function (PMF) of the discrete random variable $X$
\begin{align}
P_X(x) = P[X=x]
\end{align}
The above means, what is the probability the random variable $X$ takes the value $x$.

\subsection{Equivalent Properties to PMF}

$\forall x \in S_X$, $P_X(x) \geq 0$, means that the probability of any outcome cannot be a negative number.

$\sum_{x\in S_X} P_X(x) = 1$, means the sum of all outcomes in the range of where the probability is define is equal to 1. 

For any event $B \subset S_X$, the probability that $X$ is in the set $B$ is $P[B] = \sum_{x\in B} P_X(x)$, means that the probability that $X$ is in $B$ is the sum of all the probabilities in $B$.

\subsection{Bernoulli Distribution}

$X$ is a Bernoulli distribution ($p$) random variable if the PMF of $X$ has the following form

\begin{align}
P_X(x) = \begin{cases} 1-p, & x = 0 \\ p, & x = 1\\ 0 & \mbox{otherwise}  \end{cases}
\end{align}

The parameter $p \in (0,1)$ or simply $ 0 < p < 1$.

\subsection{Geometric Random Variables}

$X$ has a geometric ($p$) random variable if it has a distribution 
\begin{align}
P_X(x) = \begin{cases} p(1-p)^{(x-1)} & x = 1,2,\dots \\ 0 & \mbox{otherwise}\end{cases}
\end{align}

The parameter $p$ is in the range $0 < p < 1$.

\subsection{Pascal Distribution}
If you conduct an experiment with $N$ trials including up to and including $k$ successes. Then if we define the success probability of success as $p$. Then we have a pascal distribution if the PMF has the following form

\begin{align}
P_X(k,p) = {x-1 \choose k-1}p^k (1-p)^k
\end{align}

Example: find the pmf of random variables described as number of tests until we get k failed tests. 


\subsection{Discrete Uniform Random Variable}

If all the outcomes have the same probability of outcomes then the pmf can be defined as follows. 

The distribution is a function on the limits of it's sample space range. 

Let $X$ be in the range $K \leq X \leq L$. In other words, $S_X = {K, \dots, L}$. The pmf $U(K,L)$ of a uniform random variable is
\begin{align}
P[X = x]  = \frac{1}{L-K+1}
\end{align}

\subsection{Poisson Random Variable}

Models phenomenon occurring randomly in time. Each instance in time is random, but there is a known average number occurring in a given time.

\begin{align}
f(x) = \begin{cases} \frac{ \alpha^x \exp{-\alpha}}{x!}, & x=1,2,3,\dots \\ 0 & \mbox{otherwise} \end{cases}
\end{align}

An example would be the number of customers arriving to the Ministry of Transportation per time interval.

\subsection{Cumulative Distribution Function}

A cumulative distributive function is denoted as follows:
$F_X(x) = P[X \leq x]$. For any discrete random variable $X$ with range $S_X = {x_1, x_2, \dots}$ satisfying $x_1 \leq x_2 \leq \dots$ the following properties hold.

\items
\item $F(-\infty) = 0$ and $F(\infty) = 1$
\item For all $x' \geq x, F(x') \geq F(x)$
\item For $x_i \in S_X$ and let $\epsilon$ be an arbitrarily small number $F(x_i) - F(x_i - \epsilon) = P_X(x_i)$
\item $F_X(x) = F_X(x_i)$ for all $x$ such that $x_i \leq x \le x_{i+1}$
\eitems
\end{document}
